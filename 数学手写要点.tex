%\documentclass[11pt, UTF8]{ctexart}
\documentclass[11pt, a4paper]{article}
\usepackage[UTF8]{ctex}
\usepackage{amsmath, amssymb, esint, cancel} %esint提供\varoiint(闭合曲面积分号,或带圈的双重积分号), cancel用于在数学环境里画斜线
\usepackage{enumitem}%控制emumerate的缩进(但我不会用)&修改emumerate的序号样式
\usepackage[hmargin=1.25in,vmargin=1in]{geometry} %default setting in Word
\newcommand\diff{\,\mathrm{d}}
\newcommand\Newton{\mathrm{N}}

\begin{document}

\title{数学和科学手写要点\\ \centering {\Large Rules for Handwriting in Mathematics and Science}}
\author{Darron Wu}
\date{}
\maketitle


\setlength{\parindent}{1.25em} %to align this with the enumerate
\hangafter=1
\hangindent=2.5em
0. 写任何非文字部分时必须全程用力,禁止潦草,清楚就是好看。Clarity is beauty. 我们的目标是:看起来像用\LaTeX 写的。

\begin{enumerate}
%Part I. Notations
\item 数字,字母和算符要如印刷一般标准整齐,空格要清晰,即使没有横线也要底部对齐。\\
    数字:{\fontsize{13pt}{1} $\mathtt{1234567890}$}\footnote{注意:1不能向右翘脚,易与小写el混淆。2和0容易写得太小。}\\ 手写:\\[7pt]
    小写字母:{\fontsize{13pt}{1} $\mathtt{abcdefghijklmnopqrstuvwxyz}$}\footnote{我的i, l, q, z的写法有手写体的习惯。t先写勾再写横。}\\ 手写:\\[7pt]
    大写字母:{\fontsize{13pt}{1} $\mathtt{ABCDE}\mathcal{E}\mathtt{FGHIJKLMNOPQRSTUVWXYZ}$}\footnote{A为两笔。I 有三笔。L 的横要水平,保证长度。M 和N都为一笔,要写端正。U有脚。Z的转折要尖锐,避免与2混淆。}\\ 手写:\\[7pt]
    部分希腊字母:\footnote{See http://www.foundalis.com/lan/hw/grkhandw.htm for more information on handwritten letters in Greek.}{\fontsize{13pt}{1} $\alpha \beta \gamma \delta \varepsilon \zeta \eta \theta \kappa \lambda \mu \nu \xi \pi \rho \sigma \tau \varphi \chi \psi \omega \: \Gamma \Delta \Theta \Lambda \Sigma \Phi \Psi \Omega$}\footnote{$\alpha$ 的圈大小要适中,太大易与a 混淆,太小易与2混淆。$\gamma$ 的交点偏下。$\lambda$ 的左脚只有右脚的一半高。$\ nu$ 的起笔要有弯折。$\pi$ 手写时脚是竖直的。$\sigma$ 的横要平,长度为圆的一个半径。$\tau$ 为一笔连写。$\chi$用类似X的写法。$\phi$ 和$\psi$ 的竖要倾斜,$\psi$的起笔无弯折。$\Phi$ 和$\Psi$ 的竖要竖直。$\Omega$ 脚短,头圆,要写得慢一点。}\\ 手写:\\[7pt]
    常用运算符:$+ - \pm \mp \times / \div \% \sqrt{x} \ \mathbf{e}^{{x}^{a}} \log_a(b) \: \ln(b) \ \bar{a}$\\ 手写:\\[7pt]
    二元关系符:{\fontsize{12pt}{1} $= \; \ne \; \approx \; > \; < \; \le \; \ge \; \equiv \; \ll \; \gg \; : \; \propto$}\\ 手写:\\[7pt]
    三角函数:$\sin \; \arcsin \; \cos \; \arccos \; \tan \; \arctan \; \cot \; \sec \; \csc$\\ 手写:\\[7pt]
    逻辑与集合:{\fontsize{12pt}{1} $\forall \; \exists \; \Rightarrow \iff \varnothing \subset \; \supset \; \subseteq \; \supseteq \; \in \; \notin \; \ni \; \setminus \;$} $\bigcap \; \bigcup$\\ 手写:\\[7pt] % 空集一般用\varnothing 而不用\emptyset(\emptyset太瘦了)。交并集用\bigcap和\bigcup而不用\cap和\cup(它们太小了)
    几何:{\fontsize{10pt}{1} $\odot\,O \quad \triangle ABC\!\sim\!\triangle DEF \quad \cong \quad AB\!\perp CD\!\quad \angle\alpha$} \quad $\vec{a}\parallel\vec{b} \quad \vec{a}\,\cdot\,\vec{b}$ \quad {\fontsize{10pt}{1} $\overrightarrow{AB}$}\\ 手写:\\[7pt]
    微积分\footnote{手写中不区分$\diff$与$d$.}:
    \begin{displaymath}
    x \mapsto f(x) \qquad \diff z = A\Delta x + B\Delta y \qquad
    \nabla f(x, y, z) = \frac{\partial f}{\partial x}\vec{i} + \frac{\partial f}{\partial y}\vec{j} + \frac{\partial f}{\partial z}\vec{k}
    \end{displaymath}
    \begin{displaymath}
    f(x) = f(x_{0}) + f'(x_{0})(x - x_{0}) + \frac{f''(x_{0})}{2!}(x - x_{0})^{2} + \cdots + \frac{f^{n}(x_{0})}{n!}(x - x_{0})^{n} + o[(x - x_{0})^{n}]
    \end{displaymath}
    手写:\\[60pt]
    巨算符和定界符\footnote{定界符中,本条展示单竖线,绝对值符号和小括号。大括号见第\ref{brace}条,
    中括号见第\ref{bracket}条。}:
    \begin{displaymath}
    \left. \frac{\partial f}{\partial t} \right|_{t=0} \quad
    \int_0^{\pi} x f(\sin x) \diff x = \frac{\pi}{2} \int_0^{\pi} f(\sin x) \diff x \quad
    \lim_{n \to \infty}\bigg\arrowvert \frac{a_{n+1}}{a_{n}} \bigg\arrowvert = \rho
    \end{displaymath}
    \begin{displaymath}
    \iiint\limits_\Omega \left(\frac{\partial P}{\partial x} + \frac{\partial Q}{\partial y} + \frac{\partial R}{\partial z} \right) \diff v = \varoiint\limits_\Sigma P\diff y \diff z + Q\diff z \diff x + R\diff x \diff y \quad %\limits写在巨算符的上下标前可以让它们显示在巨算符的上下方(求和号式),\nolimits则会让它们显示在右上角和右下角(积分式)
    \sum_{n=0}^\infty \frac{1}{n!} x^{n} = e^{x}
    \end{displaymath}
    手写:\\[60pt]
%此行留空
\item 把科学计数法中的$\scriptstyle\times10$缩小到普通数字大小的1/2, 这样既方便写指数又可以比较清晰地区分科学计数法和乘法运算符,示例:
    \begin{displaymath}
    \frac{4v^{2}R}{G} = \frac{4 \times (3.94{\scriptstyle\times10}^{4})^{2}\mathrm{m}/\mathrm{s} \times 5.96{\scriptstyle\times10}^{9}\mathrm{m}}{6.67{\scriptstyle\times10}^{-11}\mathrm{N}\cdot \mathrm{m}^{2}/\mathrm{kg}^{2}}
    \end{displaymath}
\item \label{brace}若以$\mathbf{e}$为底的幂的指数部分太大,可以用记号$\exp\left(\langle\text{\emph{expression}}\rangle\right)$\footnote{视表达式中的括号将此处小括号变为其他括号。{给个例子?}}. 如:把$\mathbf{e}^{\frac{\ln2}{T_{1/2}}t}$ 记为
    \begin{displaymath}
    \exp\biggl( \frac{\ln2}{T_{1/2}}t \biggr)
    \end{displaymath}
\item 下标之间的相对大小与正文相同。
\item 在乘积形式的代数式中,数字在字母前面,字母按英文字母顺序排列,数字和字母放在括号前面,多个括号把简单的放在复杂的前面。带根号的代数式连根号一起放在数字和字母之后,括号之前,但$\sqrt{2}$等仅含数字的根号放在字母前面,$\pi$ 和$\mathrm{e}$后面。
\item 数字与字母、字母与字母、数字与括号、字母与括号、括号与括号之间的``×''通常简写成``$\:\cdot\:$''或省略。数字与数字之间的``×''既不能写成``$\:\cdot\:$'',也不能省略。
\item 手写分数线尽量水平。
\item 写分数的顺序:分数线,分子,分母。
\item 当分母只有几(一)个字符而分子非常长时,可以表示为分母的$-1$次幂与分子相乘。
\item 分数前只有一个负号``$-$''而无其他符号时,用小括号将其括起,避免遗漏。如\footnote{此处较好的写法是将$T_{2}$写在前面,避免使用本条所述的方法。但是在手写中本条会很实用。}:
    \begin{displaymath}
    T(r) = (-)\frac{b(T_{2}-T_{1})(r-a)}{r(b-a)} + T_{2}
    \end{displaymath}
\item 长分数线和长根号用尺。超过$3\,\mathrm{cm}$算“长”\footnote{约七个手写符号宽。}:\rule[2pt]{3cm}{0.6pt}
\item 长根号上方的横线在根号下的内容写完后再划。
\item 使用一个非约定的符号前需要声明,就像在程序里定义变量一样。
\item 同时用到两个用相同字母表示的常数时,可以用加prime($'$)的方式来区分,如:用$k$表示玻尔兹曼常数,而用$k'$表示弹性系数。
\item \label{bracket}将重复出现的因数记为$\,\alpha ,\beta ,\gamma\,$等,示例:
    \begin{displaymath}
    \text{记}\ \alpha = {\biggl[ \frac{h}{mc} \left( \frac{1}{\lambda} - \frac{1}{\lambda'} \right) + \frac{E}{mc^{2}}\biggr]}^{2}
    \end{displaymath}
\item “约去”用一条左上到右下的直线表示,示例:
    \begin{displaymath}
    \left( \frac{n}{n_{liq}} - 1 \right) \bcancel{\left( \frac{1}{R_{1}} - \frac{1}{R_{2}} \right)} =
    \frac{n - n_{liq}}{n_{liq} \bcancel{(n-1)}} \bcancel{(n-1)} \bcancel{\left( \frac{1}{R_{1}} - \frac{1}{R_{2}} \right)}
    \end{displaymath}
\item “此项为零”用一条从左下到右上的直线表示\footnote{不能在正规考试中用。},示例:
    \begin{displaymath}
    \frac{2}{n\pi}\int_{0}^{\pi} x^{2} \diff \sin(nx) =
    \frac{2}{n\pi} \left[ \cancel{\left.x^{2}\sin(nx)\right|_{0}^{\pi}} - 2\int_{0}^{\pi} x\sin(nx) \diff x \right]
    \end{displaymath}
\item 所用题号总是与题目的一致。\\
    ①②③\, 等表示方程序号。\\[0pt]
    [A][B][C]\, 等表示定理序号。\\
    $1'\  2'\  3'\ $等表示分类讨论,$(\mathsf{I})(\mathsf{II})(\mathsf{III})\,$等表示次级分类讨论。
\item 计算中需要引用某个长等式或长表达式的,可以在该式后作$(*)$标记,作为序号引用\footnote{该记号每题至多使用一次,避免混淆。}。
\item 文字与公式混排时,分数要“躺下”,如:$\frac{1}{2}$ 变为$1/2$, ``$\text{\dots according to}\ \rho(x) = \rho_{0}\exp(-x/L).$''\,列举大量分数除外,如:“将$\, \phi = 0, \frac{\pi}{4}, \frac{\pi}{2}, \frac{3\pi}{4}, \pi, \frac{5\pi}{4}, \frac{3\pi}{2}, \frac{7\pi}{4}, 2\pi$ 分别代入⑥, ……”
\item \label{units}物理计算总是带单位。Always add units in physics calculations.
\item “含单位的字母”与“带单位的数字”一起出现在代数式中时,将“带单位的数字”连单位一起括起来,如:The tension as a function of x{\scriptsize(unit: m)} is $F(x) = (392\,\Newton) + (7.70\,\Newton/\mathrm{m})x$.\footnote{\emph{Not} $F(x) = (392 + 7.70x)\Newton$, $F(x) = (392\Newton) + (7.70x)\Newton$, etc. See item \ref{units}.}


%Part II. Typesetting
\item 右边距至少为1em宽。
\item 较大和较长的公式要单独成行,开头相对文字缩进\footnote{理想的情况是居中,但手写时较难控制,故统一用缩进。}2em。
\item 有条件的,应使各算式的等号对齐。
\item 禁止在一行内连等,必须用递等式或换行另写\footnote{写到一页纸的最底部时请尤其注意这一点,该断页时就断页。},除非算式非常短\footnote{“非常短”的意思是,至少有九成把握可以在一行内写完。}。
\item 应当避免写出超过一行而需要折行的长公式。如果一定要折行的话,优先在等号之前折行,其次在加号、减号之前,再次在乘号、除号之前。其它位置应当避免折行\footnote{分数太长时,请缩小字号重写!}。示例:
    \begin{multline*}
    -\frac{{\hbar}^{2}}{2m} \left( \frac{{\partial}^{2}\psi(x,y,z)}{\partial x^{2}} + \frac{{\partial}^{2}\psi(x,y,z)}{\partial y^{2}}+ + \frac{{\partial}^{2}\psi(x,y,z)}{\partial z^{2}} \right) \\
    + U(x,y,z)\psi(x,y,z) = E\psi(x,y,z)
    \end{multline*}
\item 较高的括号\,或\,跨行的定界符\,要提前预判,或最后再画。
\item 分类讨论部分除与序号相连的段落外,缩进2em。


%Part III. Figures
\item 作图用尺和铅笔,迷你图\footnote{在A4纸上,这是指面积小于$9\,\mathrm{cm}^{2}$的图。其他情况下可以自行判断。}除外。
\item 向量减法的作图顺序\footnote{或称“向量减法的机械作图法”(相对于“动脑作图法”)。使用这种方法可以避免使用“减指被减”这条需要判断的规则,而只需按步骤操作即可,正确高效。}。如:作$\vec{a} - \vec{b}$
    \begin{enumerate}[label=(\roman*)]
    \item Draw $\vec{a}$
    \item Draw $\vec{b}\,$ from the same start as $\,\vec{a}$
    \item Connect the end of $\,\vec{b}\,$ to the end of $\,\vec{a}\,$ and add an arrow at the end\footnote{中文表述:从(刚刚画完的)$\vec{b}\,$的终点处连到$\,\vec{a}\,$的终点,并在结束的地方画一个箭头。}.
    \end{enumerate}
\item 占页面宽度超过1/2的图,不与文字混排。
\item 当一道题中涉及多张图时,要标明图的序号。


%Part IV. Others
\item “解”或“证”字单独成行\footnote{为了让接下来的内容都可以顶格写,看似浪费,实则节省空间,尤其是在纸张或栏位宽度小时。},不加冒号。
\item 对于草稿中的草稿,如:大量计算,尝试性反推,列举可用公式等\footnote{不包括用文字表达的思路,
    见第\ref{text}条。},使用{\color{blue} \emph{不同颜色}}\footnote{通常正文用黑色\,0.7mm\,水笔,“不同颜色”用蓝色\,0.7mm\,圆珠笔。},避免混乱。
\item 不要的部分在左右各画一个细长的叉来表示,不可直接划掉原来的内容。如果不要的部分超过了纸面的1/4, 那么在不要的部分的四个角上各画一个小叉,并在其顶部和底部各用尺画一条虚线。需要恢复时,在叉中间画圈即可。
\item 进行复杂运算时,说明每一步进行的操作\footnote{这对\ debug\ 和\,使用计算技巧\,都很有用。}。
\item \label{text}越难的题,越要将思路表示清楚,要好像“一个月都解不出这道题”一样详细,就像写代码必须要有注释一样。往往对一道复杂的题目,我会同时考虑四个方面:马上能求的,依赖于第一步结果的,不可能发生的\,和\,不知道能不能求的。将它们都表达出来!
\end{enumerate}




\end{document}
