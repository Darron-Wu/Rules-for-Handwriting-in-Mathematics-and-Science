%\documentclass[11pt, UTF8]{ctexart}
\documentclass[11pt, a4paper, titlepage]{article}

\usepackage[UTF8]{ctex}
\usepackage{amsmath, amssymb}
\usepackage{mathrsfs} %提供数学环境的一种字体,书法式斜体大写字母
\usepackage{esint} %在数学环境里提供\varoiint(闭合曲面积分号,或带圈的双重积分号)
\usepackage{cancel} %在数学环境里画对角斜线表示约去
\usepackage{enumitem} %修改emumerate的序号样式
\usepackage[hmargin=1.25in,vmargin=1in]{geometry} %default setting in "Word"
\usepackage{booktabs} %三线表
\usepackage{threeparttable} %三线表尾注;表后注

\newcommand{\diff}{\,\mathrm{d}} %微分号
\newcommand{\Newton}{\mathrm{N}} %单位:牛顿

\begin{document}

\title{数学和科学手写要点\\
\centering
{\Large Rules for Handwriting in Mathematics and Science\footnote{本文的标准主要是针对A4纸设定的,实际应用时,应根据纸张大小适当调整。}}}
\author{邬大容(Darron Wu)}
\date{2018/5/29}
\maketitle

\begin{enumerate}\setcounter{enumi}{-1}
\item 写任何非文字部分时必须全程用力,禁止潦草。清楚就是好看。Clarity is beauty.

\part{Notation}
\item 数字,字母和算符要如印刷一般标准整齐,空格要清晰,即使没有横线也要底部对齐    \footnote{字符的大小应以“书写感觉最自然”为宜。}。\\
    数字:
    {\fontsize{14pt}{1} $\mathtt{1234567890}$}\footnote{注意:1不能向右翘脚,易与小写el混淆。2和0容易写得太小。}\\
    手写:\\[7pt]
    小写字母:
    {\fontsize{14pt}{1} $\mathtt{abcdefghijklmnopqrstuvwxyz}$}\footnote{我的i, l, q, z的写法有手写体的习惯。m的第二个门有变矮的倾向。t先写勾再写横。}\\
    手写:\\[7pt]
    大写字母:
    {\fontsize{14pt}{1}
    $\mathtt{ABCDEFGHIJKLMNOPQRSTUVWXYZ}$}\footnote{A 为两笔。 C 的弧度要够,避免与``$\:(\:$''混淆。D的半圆必须与竖接合,不能让竖的两头突出。I有三笔。L的横要水平,保证长度。M和N都为一笔,要写端正。U有脚。Z的转折要尖锐,避免与``$\:2\:$''混淆。}\\
    手写:\\[7pt]
    简洁/艺术手写体大写字母:\\
    {\fontsize{14pt}{1}$\mathcal{ABCDEFGHIJKLMNOPQRSTUVWXYZ}$}\\
    {\fontsize{14pt}{1}$\mathscr{ABCDEFGHIJKLMNOPQRSTUVWXYZ}$}\\
    手写:\\[7pt]
    部分希腊字母\footnote{See http://www.foundalis.com/lan/hw/grkhandw.htm for more information on handwritten letters in Greek.}:
    {\fontsize{14pt}{1}
    $\alpha \beta \gamma \delta \varepsilon \zeta \eta \theta \kappa \lambda \mu \nu \xi \pi \rho \sigma \tau \varphi \chi \psi \omega \: \Gamma \Delta \Theta \Lambda \Sigma \Phi \Psi \Omega$}
    \footnote{$\alpha$ 的圈大小要适中,太大易与a 混淆,太小易与2混淆。$\gamma$ 的交点偏下。$\lambda$ 的左脚只有右脚的一半高。$\nu$ 的起笔要有弯折。$\pi$ 手写时脚是竖直的。$\sigma$ 的横要平,长度为圆的一个半径。$\tau$ 为一笔连写。$\chi$用类似X的写法。$\phi$ 和$\psi$ 的竖要倾斜,$\psi$的起笔无弯折。$\Phi$ 和$\Psi$ 的竖要竖直。$\Omega$ 脚短,头圆,要写得慢一点。}\\
    手写:\\[7pt]
    字母表示的算符\footnote{虽然在\LaTeX 中数学变量是斜体,算符是正体,但在手写特别是计算中,变量出现得比算符要多得多,所以变量用正体。那么,此类算符应该用斜体,特别是容易与变量混淆时。但应尽量使用约定的记号替代字母算符,如用$\nabla$替代$\mathrm{grad}$.}:
    {\fontsize{13pt}{1}
    $\log\: \ln\; \sin\; \arcsin\; \det\; \dim\; \lim\; \sup\; \max\; \mathrm{sgn}$\\}
    手写:\\[7pt]
    逻辑与集合:
    {\fontsize{13pt}{1}
    $\forall \; \exists \; \Rightarrow \iff \varnothing \subset \; \supset \; \subseteq \; \supseteq \; \in \; \notin \;$} $\bigcap \; \bigcup$\\%空集一般用\varnothing 而不用\emptyset(\emptyset 太瘦了)。交并集用\bigcap和\bigcup 而不用\cap和\cup(它们太小了)
    手写:\\[7pt]
    几何:
    {\fontsize{10pt}{1}
    $\odot\,O \quad \triangle ABC\!\sim\!\triangle DEF \quad \cong \quad AB\!\perp CD\!\quad \angle\alpha$} \quad $\vec{a}\parallel\vec{b} \quad \vec{a}\,\cdot\,\vec{b}$ \quad {\fontsize{10pt}{1} $\overrightarrow{AB}$}\\
    手写:\\[7pt]
    微积分\footnote{手写中不区分$\diff$与$d$.}:
    \begin{displaymath}
        x \mapsto f(x) \qquad \diff z = A\Delta x + B\Delta y \qquad
        \nabla f(x, y, z) = \frac{\partial f}{\partial x}\vec{i} + \frac{\partial f}{\partial y}\vec{j} + \frac{\partial f}{\partial z}\vec{k}
    \end{displaymath}
    \begin{displaymath}
        f(x) = f(x_{0}) + f'(x_{0})(x - x_{0}) + \frac{f''(x_{0})}{2!}(x - x_{0})^{2} + \cdots + \frac{f^{n}(x_{0})}{n!}(x - x_{0})^{n} + o[(x - x_{0})^{n}]
    \end{displaymath}
    手写:\\[60pt]
    巨算符和定界符\footnote{定界符中,本条展示单竖线,绝对值符号和小括号。大括号见第\ref{brace}条,中括号见第\ref{bracket}条。}:
    \begin{displaymath}
        \left. \frac{\partial f}{\partial t} \right|_{t=0} \quad
        \int_0^{\pi} x f(\sin x) \diff x = \frac{\pi}{2} \int_0^{\pi} f(\sin x) \diff x \quad
        \lim_{n \to \infty}\bigg\arrowvert \frac{a_{n+1}}{a_{n}} \bigg\arrowvert = \rho
    \end{displaymath}
    \begin{displaymath}
        \iiint\limits_\Omega \left(\frac{\partial P}{\partial x} + \frac{\partial Q}{\partial y} + \frac{\partial R}{\partial z} \right) \diff v = \varoiint\limits_\Sigma P\diff y \diff z + Q\diff z \diff x + R\diff x \diff y \quad %\limits写在巨算符的上下标前可以让它们显示在巨算符的上下方(求和号式),\nolimits则会让它们显示在右上角和右下角(积分式)
        \sum_{n=0}^\infty \frac{1}{n!} x^{n} = e^{x}
    \end{displaymath}
    手写:\\[60pt]

\item 把科学计数法中的$\scriptstyle\times10$缩小到普通数字大小的1/2, 这样既方便写指数又可以比较清晰地区分科学计数法和乘法运算符,示例:
    \begin{displaymath}
        \frac{4v^{2}R}{G} = \frac{4 \times (3.94{\scriptstyle\times10}^{4})^{2}\mathrm{m}/\mathrm{s} \times 5.96{\scriptstyle\times10}^{9}\mathrm{m}}{6.67{\scriptstyle\times10}^{-11}\mathrm{N}\cdot \mathrm{m}^{2}/\mathrm{kg}^{2}}
    \end{displaymath}

\item \label{brace}若以$\mathbf{e}$为底的幂的指数部分太大,可以用记号$\exp\left(\langle\text{\emph{expression}}\rangle\right)$. 如:把$\mathbf{e}^{\frac{\ln2}{T_{1/2}}t}$ 记为
    \begin{displaymath}
        \exp\biggl( \frac{\ln2}{T_{1/2}}t \biggr)
    \end{displaymath}

\item 公式中的常数用$\, \mathrm{const}\,$表示\footnote{单用一个字母c容易与变量名混淆。},如:
    \begin{displaymath}
        \int x^{a}\diff x = \frac{1}{a+1} x^{a+1} + \mathrm{const}
    \end{displaymath}

\item 理论上,下标字符之间的相对大小应与正文相同,但手写时很难做到,而且容易看不清,所以允许写成一样大的。更好的做法是通过合理的记号设置来避免多重下标。

\item 在乘积形式的代数式中,数字在字母前面,字母按英文字母顺序排列,数字和字母放在括号前面,多个括号把简单的放在复杂的前面。

\item 带根号的代数式连根号一起放在数字和字母之后,括号之前。但$\sqrt{2}$等仅含数字的根号放在字母前面,$\pi$\,和\,$\mathrm{e}$\,后面。

\item 数字与字母、字母与字母、数字与括号、字母与括号、括号与括号之间的``×''通常简写成``$\:\cdot\:$''或省略。数字与数字之间的``×''既不能写成``$\:\cdot\:$'',也不能省略。

\item 手写分数线尽量水平。

\item 写分数的顺序:分数线,分子,分母。

\item 当分母只有几(一)个字符而分子非常长时,可以表示为"分母的$-1$次幂"与分子相乘。

\item 分数前只有一个负号``$-$''而无其他符号时,用小括号将其括起,避免遗漏。如\footnote{此处较好的写法是将$T_{2}$写在前面,避免使用本条所述的方法。但是在手写中本条会很实用。}:
    \begin{displaymath}
        T(r) = (-)\frac{b(T_{2}-T_{1})(r-a)}{r(b-a)} + T_{2}
    \end{displaymath}

\item 长分数线和长根号用尺。超过$3\,\mathrm{cm}$算“长”\footnote{约我的七个手写符号宽。多长算“长”的标准是非常个人化的,我把它定义为“徒手基本可以画直的都不算长”。学过画画的人,或者姚明,可能觉得5cm算长,孩子可能觉得1cm算长。}:\rule[2pt]{3cm}{0.6pt}

\item 长根号上方的横线在根号下的内容写完后再划。

\item 使用一个非约定的符号前需要声明,就像在程序里定义变量一样。

\item 同时用到两个用相同字母表示的常数时,可以用加prime($'$)的方式来区分,如:用$k$表示玻尔兹曼常数,而用$k'$表示弹性系数。

\item \label{bracket}将重复出现的因数记为$\,\alpha ,\beta ,\gamma\,$等,示例:
    \begin{displaymath}
        \text{记}\ \alpha = {\biggl[ \frac{h}{mc} \left( \frac{1}{\lambda} - \frac{1}{\lambda'} \right) + \frac{E}{mc^{2}}\biggr]}^{2}
    \end{displaymath}

\item “约去”用一条左上到右下的直线表示,示例:
    \begin{displaymath}
        \left( \frac{n}{n_{liq}} - 1 \right) \bcancel{\left( \frac{1}{R_{1}} - \frac{1}{R_{2}} \right)} =
        \frac{n - n_{liq}}{n_{liq} \bcancel{(n-1)}} \bcancel{(n-1)} \bcancel{\left( \frac{1}{R_{1}} - \frac{1}{R_{2}} \right)}
    \end{displaymath}

\item “此项为零”用一条从左下到右上的直线表示\footnote{不能在正规考试中用。},示例:
    \begin{displaymath}[htbp!]
        \frac{2}{n\pi}\int_{0}^{\pi} x^{2} \diff \sin(nx) =
        \frac{2}{n\pi} \left[ \cancel{\left.x^{2}\sin(nx)\right|_{0}^{\pi}} - 2\int_{0}^{\pi} x\sin(nx) \diff x \right]
    \end{displaymath}

\item 所用题号总是与题目一致。

\item 序号的写法见Table \ref{number}。
    \begin{table}[h!]
        \caption{序号的写法}\label{number}
        \centering
        \begin{threeparttable}
            \begin{tabular}{lcc}
                \toprule
            类型         &首选                        &次选\tnote{1}                 \\
                \midrule
            方程         & ① \qquad ②                  & [1] \qquad\ [2]             \\
            解法         & Solution 1\qquad Solution 2 & 解法一 \qquad 解法二         \\
            证法         & Proof 1\qquad Proof 2       & 证法一 \qquad 证法二         \\
            部分\tnote{2}& Part 1\qquad Part 2         & 第一部分 \qquad 第二部分     \\
            分类讨论     & Case 1\qquad Case 2         & 情形一 \qquad 情形二         \\
            次级分类     & Subcase 1\qquad Subcase 2   & 情形一 \qquad 情形二         \\
            步骤         & Step 1\qquad Step 2         & 步骤一 \qquad\ 步骤二        \\
            定理\tnote{3}& 与参考材料相同\tnote{4}     & [A]\qquad [B]                \\
                \bottomrule
            \end{tabular}

            \begin{tablenotes}
                \footnotesize
                \item[1]仅当题中记号与首选重复时,才使用次选。
                \item[2]用于“等价”命题的证明中。
                \item[3]泛指,包括定律、公理、推论、重要结果等。
                \item[4]如果是做书后的练习,一般用书中的序号;没有序号的,用定理的名称;两个都没有的,标明页数和行数。其他材料同理。
            \end{tablenotes}

        \end{threeparttable}
    \end{table}

\item 计算中需要引用某个单独成行的等式或表达式的,可以在该式后作$(*)$标记,作为序号引用
    \footnote{该记号每题至多使用一次,避免混淆。}。

\item 文字与公式混排时,分数要“躺下”,如:在``$\text{\dots according to}\ \rho(x) = \rho_{0}\mathrm{e}^{-x/L}$''中,使用$\mathrm{e}^{-x/L}$而不是$\mathrm{e}^{-\frac{x}{L}}$.\footnote{对于更复杂的以$\mathrm{e}$为底的幂,参见第\ref{brace}条。}\\
    例外是列举大量分数时,如:“将$\, \phi = 0, \frac{\pi}{4}, \frac{\pi}{2}, \frac{3\pi}{4}, \pi, \frac{5\pi}{4}, \frac{3\pi}{2}, \frac{7\pi}{4}, 2\pi$分别代入⑥, ……”

\item\label{units}物理计算总是带单位。Always add units in physics calculations.

\item “含单位的字母”与“带单位的数字”一起出现在代数式中时,将“带单位的数字”连单位一起括起来,如:The tension as a function of x{\scriptsize(unit: m)} is $F(x) = (392\,\Newton) + (7.70\,\Newton/\mathrm{m})x$.\footnote{\emph{Not} $F(x) = (392 + 7.70x)\Newton$, or, $F(x) = (392\Newton) + (7.70x)\Newton$, etc. 见第\ref{units}条。事实上,应当尽量避免变量与单位混写的情况。}

\item 一个物理量用“不含单位的变量”表示时,将单位前的表达式括起来,如:$v = (40n + 10) \, \mathrm{m/s},\, n = 0,1,2,...$

\item 在省略号之前和之后都加上逗号。Put commas before and after ellipses, such as $(P_{1},\dots, P_{n})$.

\item 当省略号夹在符号之间时,它们同高。具体来说,夹在逗号之间就底部对齐;夹在``$+$''或``$<$''之间就居中。When ellipses are between commas they belong on the same level as the commas; when ellipses are bracketed by symbols such as ``$+$'' or ``$<$'', they should be at mid-level.




\part{Typeset}
\item Symbols in different formulas must be separated by words.
    \begin{quote}
    Bad: Consider $S_{q}$, $q < p$.\\
    Good: Consider $S_{q}$, where $q < p$.
    \end{quote}

\item Don't start a sentence with a symbol.
    \begin{quote}
    Bad: $x_{n} - a$ has $n$ distinct zeroes.\\
    Good: The polynomial $x_{n} - a$ has $n$ distinct zeroes.
    \end{quote}

\item Don't use the symbols $\therefore\, , \ \forall\, , \ \exists, \ \Rightarrow\, , \ \in$, etc.; replace them by the corresponding words\footnote{Except in works on logic, of course.}.

\item Should the first word after a colon be capitalized? Yes, if the phrase following the colon is a full sentence; No, if it is a sentence fragment.

\item 对关于一本书或一门课的草稿纸编号,置于右上角\footnote{有编号的草稿纸不可折叠,并且使用时需要较严格地遵守本规定。对于不编号的草稿纸,本规定仅作参考之用。}。编号规则:
    \begin{enumerate}[label=(\Roman*)]
        \item 封面:0。单开一页。
        \item 第$\, 200n+1$页至第$\, 200n+100$页($n\in \mathrm{N}$):新开一张纸并编号。
        \item 其余页:在已编号的纸反面,在原编号上加$100$.
    \end{enumerate}

\item 默认左边距为零.

\item 默认右边距为2em\footnote{这是为了提醒自己,数学表达式会溢出时,要换行。对单词同样有用。对汉字几乎没有影响。}.

\item “证(Proof)”单独成行\footnote{为了让接下来的内容都可以顶格写,看似浪费,实则节省空间,尤其是在纸张或栏位宽度小时。},不加标点\footnote{因为单独成行已经起足够清晰了,不再需要通过标点来分割或强调,就好像我们从来不会在行间公式后面加标点一样。}。“解(Solution)”、“讨论(Discussion)”、“评论(Remark)”等同理。

\item 分类讨论部分除与序号相连的段落外,缩进2em.

\item 较大、较长或需要强调的算式要单独成行。\\
    较大:高度大于汉字高度1.3倍的。\\
    较长:长度达10em的,或可能中途断行的。\\

\item 单独成行的算式,左边相对文字缩进\footnote{理想的情况是居中,但手写时较难控制,故统一用缩进。}。缩进的大小一般为2em. 当算式长度接近一行时,以第\ref{newline}条为准,不缩进或负缩进也是允许的。

\item 单个分数长度超出一行时,需要通过拆分、缩小字号、化简等方法使其可被完整地写在一行内。

\item \label{newline}应当避免写出需要折行的长公式。如果一定要折行的话,优先在等号之前折行,其次在加号、减号之前,再次在乘号、除号之前。其它位置应当避免折行。示例:
    \begin{multline*}
        -\frac{{\hbar}^{2}}{2m} \left( \frac{{\partial}^{2}\psi(x,y,z)}{\partial x^{2}} + \frac{{\partial}^{2}\psi(x,y,z)}{\partial y^{2}}+ + \frac{{\partial}^{2}\psi(x,y,z)}{\partial z^{2}} \right) \\
        + U(x,y,z)\psi(x,y,z) = E\psi(x,y,z)
    \end{multline*}

\item 有条件的,应使各算式的等号对齐。

\item 禁止在一行内连等,必须用递等式或换行另写\footnote{写到一页纸的最底部时请尤其注意这一点,该断页时就断页。},除非算式非常短\footnote{“非常短”的意思是,至少有九成把握可以在一行内写完。}。

\item 较高的括号\,或\,跨行的定界符\,要提前预判,或最后再画。

\item 表示“证毕”的方块放在证明末行,右对齐\footnote{尽管这个符号有一些排版价值,但手写这个主要是为其自我鼓励作用。}。

\item 两节之间设分割实线。跨页时省略此线。





\part{Figure}
\item 作图用尺和铅笔,迷你图\footnote{在A4纸上,这是指面积小于$9\,\mathrm{cm}^{2}$的图。其他情况下可以自行判断。}除外。

\item 向量减法的作图顺序\footnote{或称“向量减法的机械作图法”(相对于“动脑作图法”)。使用这种方法可以避免使用“减指被减”这条需要判断的规则,而只需按步骤操作即可,正确高效。}。如:作$\vec{a} - \vec{b}$
    \begin{enumerate}[label=(\roman*)]
        \item Draw $\vec{a}$
        \item Draw $\vec{b}\,$ from the same start as $\,\vec{a}$
        \item Connect the end of $\,\vec{b}\,$ to the end of $\,\vec{a}\,$ and add an arrow at the end\footnote{中文表述:从(刚刚画完的)$\vec{b}\,$的终点处连到$\,\vec{a}\,$的终点,并在抬笔的地方画一个箭头。}.
    \end{enumerate}

\item 物理中,用从圆心出发的箭头标示半径长度\footnote{这是为了避免与实际存在的细线、细杆等物体混淆。}。

\item 占页面宽度超过1/2的图,不与文字混排。

\item 当一道题中涉及多张图时,要标明图的序号。




\part{Miscellany}
\item 作差应使用和结论相同的结构,即,比较$a$和$b$的大小时,应比较$a-b$与$0$的大小,而不是$b-a$.

\item Underline every occurrence of the symbol $\,O$, to distinguish them from 0 (zero). Also, underline the letter $\,a\,$ when it appears as a symbol in the running text, to distinguish it from the indefinite article.

\item Text preceding displayed equations are not followed by any special punctuation, especially colons.

\item Commas and periods should be placed inside quotation marks, but other punctuation, like colons and question marks, stay outside the quotation marks unless they are part of the quotation.

\item Numbers smaller than ten should be spelled out when used as adjectives and numbers, but not when used as numerals.
    \begin{quote}
    Bad: The method requires 2 passes.\\
    Good: The method requires two passes.\\
    Bad: The number of solutions is either 2 or 3.\\
    Good: The number of solutions is either two or three.\\
    Bad: The leftmost ``two'' in the sequence was changed to a ``one''.\\
    Good: The leftmost 2 in the sequence was changed to a 1.
    \end{quote}

\item Don't name the elements of a set unless necessary. Then you can refer to elements $x$ and $y$ in your subsequent discussion, without needing subscripts; or you can refer to $x_{1}$ and $x_{2}$ as specified elements. Violation of this rule calls for trouble. Starting out with a definition like ``Let $X = \{x_{1},\dots ,x_{n}\}$'', you'll need to be speaking of elements $x_{i}$ and $x_{j}$ all the time. Moreover, if you're going to need subsets of $X$, the subset will have to defined as, say, $\{x_{{i}_{1}},\dots ,x_{{i}_{n}}\}$. This involves subscripted subscripts therefore should be avoided.

\item 较长的论述中,可用斜体或下划线来强调文字\footnote{注意,是强调“文字”,不是“字符”。一般地,字符在手写中不能被强调。}。

\item \label{color}对于草稿中的草稿,如:大量计算,尝试性反推,列举可用公式等\footnote{不包括用文字表达的思路,见第\ref{text}条。},使用{\fontsize{11.5pt}{1}\color{blue}\emph{不同颜色}}\footnote{通常正文用黑色\,0.7mm\, 水笔,“不同颜色”用蓝色\,0.7mm\,圆珠笔。},避免混乱。

\item 不要的部分在左右各画一个细长的叉来表示,不可直接划掉原来的内容。如果不要的部分超过了纸面的1/4, 那么在不要的部分的四个角上各画一个小叉,并在其顶部和底部各用尺画一条虚线。需要恢复时,在叉中间画圈即可。

\item 进行复杂运算时,说明每一步进行的操作\footnote{这对\ debug\ 和\,使用计算技巧\,都很有用。}。

\item 当字母有取值范围时,必须给出\emph{完整}的范围,尤其是在次级分类讨论中。

\item 不属于解答过程的元素放在虚线方框内。如:解法的注释、相关内容的标注、对题目类型的评论\footnote{注意本条与第
    \ref{color}条的区别。}。

\item 卡住时,表述卡住的原因,越详细越好\footnote{So often, the only trouble one has is what the trouble is. 同学来问我题目的时候,往往,我只用一句话就可以解决:“你的问题是什么?”}。

\item \label{text}越难的题,越要将思路表示清楚,要好像“一个月都解不出这道题”一样详细,就像写代码必须要有注释一样\footnote{线性的解题过程是整理得到的,实际的解题过程一般是发散的。对一道复杂的题目,我会同时考虑四个方面:“马上能得到的”,“依赖于第一步结果的”,“不可能发生的”和“从结论可反推的”。这些都是有用的线索,必须写下来。}。

\item 做错的题,如果看过解答仍无法解决,要做记号并整理到清单上。
\end{enumerate}

\end{document}
